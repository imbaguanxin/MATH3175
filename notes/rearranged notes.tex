\documentclass[12pt]{article}  
\usepackage{graphicx}
\usepackage{geometry}   %设置页边距的宏包
\usepackage{algpseudocode}
\usepackage{comment}
\usepackage{amsmath, amssymb, amsthm}
\usepackage{enumerate}
\usepackage{enumitem}
\usepackage{framed}
\usepackage{verbatim}
\usepackage{microtype}
\usepackage{kpfonts}
\usepackage{multicol}
\usepackage{amsfonts}
\usepackage{array}
\usepackage{color}
\usepackage{pgf,tikz}
\usepackage{mathtools}
\usetikzlibrary{automata, positioning, arrows}
\usepackage{wrapfig}
\newcommand{\solu}{{\color{blue} Solution:}}
\newcommand{\theo}{{\color{blue} Theorem: $\ $}}
\newcommand{\defi}{{\color{blue} Definition: $\ $}}
\newcommand{\recall}{{\color{blue} Recall: $\ $}}
\newcommand{\exe}{{\color{green} Example: $\ $}}
\newcommand{\prop}{{\color{blue} Prop: $\ $}}
\newcommand{\rem}{{\color{blue} Remark: $\ $}}
\newcommand{\coro}{{\color{blue} Corollary: $\ $}}



\newcommand{\hw}{{\color{red} Homework: $\ $}}
\newcommand{\overbar}[1]{\mkern 1.5mu\overline{\mkern-1.5mu#1\mkern-1.5mu}\mkern 1.5mu}
\newcommand{\Ib}{\mathbf{I}}
\newcommand{\Pb}{\mathbf{P}}
\newcommand{\Qb}{\mathbf{Q}}
\newcommand{\Rb}{\mathbf{R}}
\newcommand{\Nb}{\mathbf{N}}
\newcommand{\Fb}{\mathbf{F}}
\newcommand{\Z}{\mathbf{Z}}
\newcommand{\Lap}{\mathcal{L}}
\newcommand{\Zplus}{\mathbf{Z}^+}
\geometry{left=2cm,right=2cm,top=1.5cm,bottom=2cm}  %设置 上、左、下、右 页边距

\title{MATH 3175 quiz3 review}
\date{}
\author{Xin Guan}

\begin{document}
    \begin{itemize}
        \item \textbf{Bijection}: \\
        Function $f: X \rightarrow Y$ is bijection if $f$ is both surjection(on to) and injection (one to one)
        \textbf{Proposition}:
        \begin{enumerate}
            \item $f: X \rightarrow Y$ is bijection $\Leftrightarrow$\\
            $\exists g: Y \rightarrow X$ s.t. $g \circ f = id_x, f \circ g = id_y$ ($id_x \rightarrow$ identity)
            \item Composition Properties:\\
            $\circ$ Composition of two injective functions is injective. \\
            $\circ$ Composition of two surjective functions is surjective.\\
            $\circ$ Composition of two bijective functions is bijective.
        \end{enumerate}
        \item \textbf{Permutation:}\\
        Permutation on set $X$ is a bijection $f : X \rightarrow X$\\
        If $X = \{1,2,\dots,n\}$ then, $S_n :=$ \{all permutation on $X$\} 
        \textbf{Proposition:}
        \begin{enumerate}
            \item if $f: X \rightarrow X$ is a permutation then $\exists f^{-1}: X \rightarrow X$ which is also permutation.
            \item composition of two permutation is again a permutation.
        \end{enumerate}
        \item \textbf{Group 5 Rules}:
        \begin{enumerate}
            \item Closed under binary operation
            \item associative: (ab)c = a(bc)
            \item identity: $\exists e \in G, ea = ae = a \forall a \in G$
            \item inverse: $\forall a \in G, \exists! a^{-1} s.t. a^{-1}a = aa^{-1} = e$
            \item commutative $a,b \in G, ab = ba$.
        \end{enumerate}
        1,2: semigroup\\
        1,2,3: monoid\\
        1,2,3,4: group\\
        1,2,3,4,5: Abelian group
        \item \textbf{Equivalence Relation:}\\
        Operation $\sim$ in Group $G$ is equivalence if 
        \begin{enumerate}
            \item Reflective: $g \sim g, \forall g \in G$
            \item Symmetry: $g \sim g' \Rightarrow g' \sim g, \forall g,g' \in G$
            \item transitive: $x \sim y, y \sim z \Rightarrow x \sim z \forall x,y,z$
        \end{enumerate}
        \item \textbf{Subgroup}: $H$ is a subgroup of $G$ if
        \begin{itemize}
            \item $H \subseteq G$
            \item $H$ is a group
        \end{itemize}
        \textbf{CHECK a SUBGROUP:}
        \begin{itemize}
            \item $H \subseteq G$ (subset)
            \item $e \in H$ (non empty)
            \item $\forall a,b \in H, ab \in H$ (closed)
            \item $\forall a \in H, a^{-1} \in H$
        \end{itemize}
        Proper subgroup: subgroup $H$ that is not $H \ne G$
        \item \textbf{Order}:\\
        Order of a group: $|G|$ = \# of elements in the group. If a group is infinite, then the order is $\infty$\\
        Order of an element: $g\in G, |g| = \textbf{smallest positive integer }n, \ s.t. \ x^n = e$\\
        \textbf{Propositions:}
        \begin{itemize}
            \item Let $g \in G$, $|<g>| = |g|$
            \item If $H$ is a subgroup of $G$ then $|H| \ | \ |G|$. If $x \in G$, then $|x| \ | \ |G|$
        \end{itemize}
        \item $<x> := \{\ x^n \ | \ n \in \mathbb{Z}\}$
        \item \textbf{Conjugate}: $x,g\in G$, conjugate of $x$ by $g$: $gxg^{-1}$\\
        Conjugate class of x:= \{$gxg^{-1} \ | \ \forall g \in G$\}
        \item \textbf{ISOMORPHISMS of GROUP}: a function $f : G \rightarrow G'$ is called isomorphism if:
        \begin{enumerate}
            \item $f(xy) = f(x)f(y)$
            \item $f$ is one to one (injective)
            \item $f$ is onto (surjective)
        \end{enumerate}
        $G \cong G'$ (group isomorphisim): $\exists f:G \rightarrow G'$ that is isomorphic. Then $|G| = |G'|$.\\
        \textbf{Propositions:}
        \begin{itemize}
            \item \ Suppose $G\cong  G'$ Then $G$ is abelian $\Leftrightarrow$ $G'$ is abelian. (which implies that abelian group and non abelian group is not isomorphic)
            \item if $G$ and $G'$ are cyclic and $|G| = |G'|$ then $G\cong G'$ 
            \item Let $G = (Z_n, +_n) = \{[0], [1] \cdots, [n-1]\}$, $G' = (Z_n, +_n) = (\{0,1,2,\cdots, n-1\}, +_n)$
            Then $G\cong G'$ and the isomorphisim can be take $[x]_n \rightarrow x$
        \end{itemize}
        \item \textbf{Cyclic}: $\exists a \in G, \ s.t. \ <a> = G$ such $a$ is called a generator.
        \item \textbf{Center of Group}:\\
        Center of a Group $G: Z(G):= \{Z\in G | gz = zg, \ \forall g \in G\}$\\
        \textbf{Proposition:}
        \begin{enumerate}
            \item $Z(G)$ is a subgroup of $G$.
            \item If $G$ is abelian, then $Z(G) = G$
        \end{enumerate}
        \item \textbf{External direct product of Groups:}\\
        Group $G, H$, Define $G \times H :=\{(x,y) \ | \ x\in G, y \in H\}$\\
        $(x_1,y_1)(x_2, y_2) = (x_1x_2, y_1y_2)$\\
        \textbf{Proposition:}
        \begin{enumerate}
            \item $e_{G\times H} = (e_G, e_H)$
            \item $(x,y)^{-1} = (x^{-1}, y^{-1})$
            \item $|(x,y)| = LCM(|x|, |y|)$
        \end{enumerate}
        \item \textbf{Internal product of groups:}\\
        Group $G$ has subgroup $H, K$. Defind $HK:=\{xy | x\in H, y \in K\}$\\
        \textbf{NOTE:} $HK$ is not always a subgroup.\\
        \textbf{Proposition:}
        \begin{enumerate}
            \item $H,K$ are subgroup of $G$. \\
            Suppose $x^{-1}yx\in K, \forall x\in H, y \in K$ Then $HK$ is a subgroup of $G$.\\
            \textbf{Corollary}: $H,K$ are subgroup of abelien group $G$, then $HK$ is a subgroup of $G$.
        \end{enumerate}
        \item \textbf{Group Homomorphisms:}\\
        $f: G \rightarrow G'$ if $f(xy) = f(x)f(y) \forall x,y \in G$\\
        Compared to isomorphism, we don't need bijection.
        \item \textbf{Kernal and Image:}\\
        $f: G \rightarrow G'$, Define:\\
        Kerf $:=\{g \in G \ | \ f(g) = e_G'\}$\\
        Imf $:= \{y \in G' \ | \ \exists x \in G, s.t. \ f(x) = y\} \equiv \{f(x) \ | \ x\in G\}$\\
        \textbf{Lemma:}\\
        $f: G \rightarrow G'$ be a group homomorphism.
        \begin{enumerate}
            \item $f(e_G) = e_{G'}$
            \item $f(a^n) = (f(a))^n, \forall n >0, n \in \mathbb{Z}$
            \item $f(a^{-1}) = (f(a))^{-1}$
            \item From 2,3 we can conclude: $f(a^n) = (f(a))^n$
        \end{enumerate}
        \textbf{Proposition:}
        \begin{enumerate}
            \item $f: G \rightarrow G'$ be group homomorphism:
            \begin{itemize}
                \item kerf is a subgroup of $G$
                \item Imf is a subgroup of $G'$
            \end{itemize}
            \item If $G = <a>$ i.e. $G$ is a cyclic group. Then, it is enough to define homomorphism $f:G\rightarrow G'$ on $a$ and extend to all $a^n$.
            \item $f: G\rightarrow G'$ be a group homomorphism, then $|f(a)| \ | \ |a|$
        \end{enumerate}
        \item \textbf{Left Coset and Right Coset:}\\
        \defi \\
        Let $G$ be a group , let $H$ be a subgroup of $G$.\\
        Left coset of $H$ in $G$: $aH := \{ah \ | \ h \in G\}$\\
        Right coset of $H$ in $G$: $Ha := \{ha \ | \ h \in G\}$
        \textbf{Proposition:}
        \begin{itemize}
            \item $aH = H$ iff $a \in H$
            \item \begin{tabular}{ccc}
                $aH = bH$ & $\Leftrightarrow$ & $a\in bH$ \\
                  & $\Leftrightarrow$ & $b\in aH$ \\
                  & $\Leftrightarrow$ & $a^{-1}b\in H$ \\
                  & $\Leftrightarrow$ & $b^{-1}a\in H$
            \end{tabular}
            \item $aH \cup bH$ = $\emptyset$ or $aH = bH$.\\
            Only two posibilities. When it is $\emptyset$, properties above fails. When it is not $\emptyset$ the only posibility is that $aH = bH$ and above properties holds.
            \item $G = \sqcup aH$ (disjoint union of left cosets.)\\
            Taking elements inside the set $H$ won't generate new cosets. Only taking elements outside the set would generate new cosets.
            \item $H < G, |aH| = |H|, \forall a \in G$
        \end{itemize}
        
        \defi $H < G$, $[G:H]:=$ \# of left cosets of $H$ in $G$\\
        \textbf{Properties:} $|G| = [G:H] \cdot |H| \Leftrightarrow [G:H] = |G|/|H|$\\
        In general, $aH \ne Ha$, sometimes they are the same.
        \item \textbf{Normal Subgroup:}\\
        \defi $H <G, H$ is normal subgroup $\Leftrightarrow H \triangleleft G$ if $aH = Ha, \forall a \in G$.\\
        \theo $H < G$, the following are equivalent:
        \begin{itemize}
            \item $H \triangleleft G$
            \item $aH = Ha, \forall a \in G$
            \item $aHa^{-1} \subseteq H, \forall a \in G$
            \item $aHa^{-1} = H, \forall a \in G$
        \end{itemize}
        \prop If $G$ is an abelian group, then every subgroup of $G$ is normal.
    \item \textbf{Symmetric Group:}\\
        \defi transposition is an element $\tau_{ij} = (ij)$ (permutation of length 2)\\
        \prop Every permutation seauence $b \in S_n$ can be written as a product of transpositions.
        \begin{itemize}
            \item Step 1: write the permutation in disjoint cycles.
            \item Step 2: write each cycle as a product of transpositions.
        \end{itemize}
        \exe $(1346)(13)(14)(16) = (3461) \Rightarrow (16)(14)(13) = (1346)$
    \item \textbf{Sign of permutation:}
        \defi the sign of permutation $\sigma$ is the parity of the number of transpositions in any decompositions.\\
        To conclude: length of cycle is even $\Rightarrow$ parity odd; length of cycle is odd $\Rightarrow$ parity even.
        \prop even$\cdot$ even = even; even$\cdot$odd = odd; odd$\cdot$even = odd; odd$\cdot$odd = even
    \item \textbf{Quotient Group:}\\
    $H \triangleleft G$ then $G/H = \{aH \ | \ a \in G\}$. \\
    $G/H$ is a group under operation: $(aH)(bH) = abH$. identity: $e_{G/H} = eH = H$. inverse: $(aH)^{-1} = a^{-1}H$
    \item \textbf{Isomorphism Theorem:}
    \begin{enumerate}
        \item Let $f: G \rightarrow G'$ be a group homomorphism. Then $G/kerf \cong imf$
        \begin{itemize}
            \item Let $H < G$, let $i: H \rightarrow G$ be $i(x) = x$.\\
            Then, 1. i is a gorup homomorphism.\\
            2. i is one to one.\\
            3. $ker(i) = e$\\
            4. $im(i) = H$.
            \item Let $H \triangleleft G$. Let $\pi: G \rightarrow G/H$ be given as $\Pi(x) = xH$\\
            Then, 1. $\pi$ is a group homomorphism.\\
            2. $ker(\pi) = H$\\
            3. $\pi$ is onto \\
            4. $Im(\pi) = G/H$.
            \item \textbf{Theorem:} $f: G \rightarrow G'$, $f$ is a group homomorphism.
            \begin{enumerate}
                \item $ker(f) \triangleleft G$
                \item $G/kerf$ is a group.
                \item $Imf \le G'$, $Imf$ is a subgroup.
                \item $f: G \rightarrow G'$ becomes:\\
                $G \xrightarrow{\pi} G/kerf  \xrightarrow{\overbar{f}} Imf \xrightarrow{i'} G'$ , where $\overbar{f}(a\cdot kerf) = f(a)$\\
                Then $f = i'\overbar{f}\pi$, i.e. any $f$ can be write in composition of $i\overbar{f}\pi$
            \end{enumerate}
        \end{itemize}
        \item Let $H, N < G$, suppose $N \triangleleft G$, then $(H\cdot N) / N \cong H / (H \cup N)$
        \item Let $K, H \triangleleft G$ (normal subgroup), suppose $K \subset H$. Then, $(G/K)\ / \ (H/K) \cong G / H$
    \end{enumerate}
    \end{itemize}
\end{document}