\documentclass[12pt]{article}  
\usepackage{graphicx}
\usepackage{geometry}   %设置页边距的宏包
\usepackage{algpseudocode}
\usepackage{comment}
\usepackage{amsmath, amssymb, amsthm}
\usepackage{enumerate}
\usepackage{enumitem}
\usepackage{framed}
\usepackage{verbatim}
\usepackage{microtype}
\usepackage{kpfonts}
\usepackage{multicol}
\usepackage{amsfonts}
\usepackage{array}
\usepackage{color}
\usepackage{pgf,tikz}
\usetikzlibrary{automata, positioning, arrows}
\usepackage{wrapfig}
\newcommand{\solu}{{\color{blue} Solution:}}
\newcommand{\theo}{{\color{blue} Theorem: $\ $}}
\newcommand{\defi}{{\color{blue} Definition: $\ $}}
\newcommand{\recall}{{\color{blue} Recall: $\ $}}
\newcommand{\exe}{{\color{green} Example: $\ $}}
\newcommand{\prop}{{\color{blue} Prop: $\ $}}
\newcommand{\rem}{{\color{blue} Remark: $\ $}}
\newcommand{\coro}{{\color{blue} Corollary: $\ $}}



\newcommand{\hw}{{\color{red} Homework: $\ $}}
\newcommand{\overbar}[1]{\mkern 1.5mu\overline{\mkern-1.5mu#1\mkern-1.5mu}\mkern 1.5mu}
\newcommand{\Ib}{\mathbf{I}}
\newcommand{\Pb}{\mathbf{P}}
\newcommand{\Qb}{\mathbf{Q}}
\newcommand{\Rb}{\mathbf{R}}
\newcommand{\Nb}{\mathbf{N}}
\newcommand{\Fb}{\mathbf{F}}
\newcommand{\Z}{\mathbf{Z}}
\newcommand{\Lap}{\mathcal{L}}
\newcommand{\Zplus}{\mathbf{Z}^+}
\geometry{left=2cm,right=2cm,top=1.5cm,bottom=2cm}  %设置 上、左、下、右 页边距

% \tikzset{
%     ->, % makes the edges directed
%     >=stealth, % makes the arrow heads bold
%     node distance=2.5cm, % specifies the minimum distance between two nodes. Change if necessary.
%     every state/.style={thick, fill=gray!10}, % sets the properties for each ’state’ node
%     initial text=$ $, % sets the text that appears on the start arrow
% } 

\title{MATH 3175 Notes}
\date{}
\author{Xin Guan}


\begin{document}  
\maketitle
\begin{enumerate}
    \item 01.22
    \begin{enumerate}
        \item \defi 7.1
        
        Function $f: X \rightarrow Y$ is bijection if $f$ is both surjection(on to) and injection (one to one)

        \item \theo 7.2 
        
        $f: X \rightarrow Y$ is bijection $\Leftrightarrow$\\
        $\exists g: Y \rightarrow X$ s.t. $g \circ f = id_x, f \circ g = id_y$ ($id_x$ means identity)

        Such $g$ is called the inverse of f. Denoted by $f^{-1}$

        \item \recall 
        
        $\circ$ Composition of two injective functions is injective. 

        $\circ$ Composition of two surjective functions is surjective.

        $\circ$ Composition of two bijective functions is bijective.

        \item \defi 7.4 Permutation:
        
        Permutation on set $X$ is a bijection $f : X \rightarrow X$

        \item prop 7.5
        \begin{enumerate}
            \item if $f: X \rightarrow X$ is a permutation then $\exists f^{-1}: X \rightarrow X$ which is also permutation.
            \item composition of two permutation is again a permutation.
        \end{enumerate}

        \item \defi 7.6
        
        if $X = \{1,2,\dots,n\}$ then, $S_n :=$ \{all permutation on $X$\} 

        \item EX 7.7
        
        $\alpha = \begin{pmatrix}
            1 & 2 & 3 & 4 & 5 \\
            3 & 4 & 1 & 2 & 5
        \end{pmatrix}$\\
        $\beta = \begin{pmatrix}
            1 & 2 & 3 & 4 & 5 \\
            5 & 1 & 4 & 3 & 2
        \end{pmatrix}$

        Find $\alpha\beta$ (composition of $\alpha$ and $\beta$), $\alpha^{-1}$

        \solu 

        $(\alpha\beta)(1) = \alpha(\beta(1)) = \alpha(5) = 5$\\
        $(\alpha\beta)(2) = \alpha(\beta(2)) = \alpha(1) = 3$\\
        $(\alpha\beta)(3) = \alpha(\beta(3)) = \alpha(4) = 2$\\
        $(\alpha\beta)(4) = \alpha(\beta(4)) = \alpha(5) = 1$\\
        $(\alpha\beta)(5) = \alpha(\beta(5)) = \alpha(2) = 4$

        Then, $\alpha\beta = \begin{pmatrix}
            1 & 2 & 3 & 4 & 5 \\
            5 & 3 & 2 & 1 & 4
        \end{pmatrix}$

        $\alpha^{-1} = \begin{pmatrix}
            3 & 4 & 1 & 2 & 5 \\
            1 & 2 & 3 & 4 & 5
        \end{pmatrix}$\\
        rearrange:\\
        $\alpha^{-1} = \begin{pmatrix}
            1 & 2 & 3 & 4 & 5 \\
            3 & 4 & 1 & 2 & 5
        \end{pmatrix}$

        \item \hw 2.1 9(b) 
        
        $g : \Z_8 \Rightarrow \Z_12, g([x]_8) = [6x]_12$ show that g is well defined.
        
        \solu

        \begin{proof}
        Suppose $[x]_8 = [x']_8$, WTS $g([x]_8) = g([x']_8)$

        Let $[x]_8 = [x']_8 \\ \Rightarrow x \equiv x' (mod \ 8) \\ \Rightarrow 8|(x - x') \\ \Rightarrow x - x' = 8*q $ for some $q \in \Z$ \\
        $x = 8\cdot q + x'$

        By definition of $g$, $g([x]_8) = [6x]_{12}$\\
        Then, $g([x]_8) = [6(8q+x')]_{12} = [48q + 6x']_{12}, g([x']_8) = [6x']_{12}$\\
        WTS $[48q + 6x']_{12} = [6x']_{12}$\\
        Enough to show: $12 | (48q + 6x' - 6x')$\\
        Since $48q + 6x' - 6x' = 48 q = 12 \cdot 4 \cdot q$\\
        $\Rightarrow 12 | 12 \cdot 4 \cdot q$\\
        $\Rightarrow 12 | (48q + 6x' - 6x')$ \\
        $\Rightarrow g([x]_8) = g([x']_8)$
        \end{proof}
    \end{enumerate}
    \item 01.23
    \begin{enumerate}
        \item \recall
    
        DEF: Permutation on set X is a bijection $f: X \rightarrow X$

        NOTE: $S_x$ = \{permutation on X\}, $S_n$ = \{permutation on \{1,2,3, ... n\}\}

        PROPERTIES: 
        
        composition of permutation is again a permutation.

        identity map: $id: X \rightarrow X (id(x) = x)$ is a permutation.

        each permutation $f$ there is an inverse $f^{-1}$ such that $f \circ f^{-1} = id, f^{-1} \circ f = id$.

        \item \defi 8.1 Disjoint cycle decomposition
    
        Suppose $\alpha = \begin{pmatrix}
            1 & 2 & 3 & 4 & 5 & 6 & 7 & 8 \\
            3 & 4 & 8 & 1 & 2 & 6 & 7 & 5
        \end{pmatrix}$ \\= (1 3 8 5 2 4)(6)(7) or (1 3 8 5 2 4) (in cycle notation)
    
        \item \defi 8.2 
    
        2-cycle $\rightarrow$ (i j) $i\ne j$ \\
        3-cycle $\rightarrow$ (i j k) i,j,k distinct \\
        r-cycle $\rightarrow$ ($i_1, i_2, \dots, i_r$),  $i_1, i_2, \dots, i_r$ distinct

        \item \exe 8.3
    
        $\alpha = (142), \beta = (13)$
        $\alpha \rightarrow$ 3-cycle, $\beta \rightarrow$ 2-cycle.

        \item identity permutation in $S_n$
    
        \begin{enumerate}
            \item (1)(2)$\dots$(n)
            
            \item fixes $\forall i$
            
            \item 1-cycle $(i)$ fixes $i$
            
            \item often we do not note 1-cycle: $\alpha = (142) = (142)(3)$
            
            \item id = (1) = (1)(2)$\dots$(n)
        \end{enumerate}

        \item \exe 8.5 multiplication of permutation
        
        $\alpha = (142), \beta = (13), \in S_4$

        compute - write as a product of disjoint cycles (same as \exe 7.7 with new notation)

        $\alpha\beta = (142)(13) = (1342)$

        HOWTO: $\beta: 1 \rightarrow 3$, then $\alpha: 3 \rightarrow 3$, then (13) now.\\
        $\beta: 3 \rightarrow 1$, then $\alpha: 1 \rightarrow 4$, then (134) now.\\
        $\beta: 4 \rightarrow 4$, then $\alpha: 4 \rightarrow 2$, then (1342).

        Similarly: $\beta\alpha = (13)(142) = (1423)$

        \item \rem 8.6
        In general $\alpha\beta \ne \beta\alpha$\\
        if $\alpha, \beta$ are disjoint then $\alpha\beta  = \beta\alpha$
        

        \item \defi 8.7 
        
        Order of permutation $\alpha$ is the smallest positive integer $n$ such that $\alpha^n = (1)$ where $\alpha^n = \alpha \alpha \dots \alpha$ (there are n $\alpha$'s)

        \item \exe 8.8 

        $\alpha = (142)$

        $\alpha^2 = \alpha \alpha = (142)(142) = (124)$

        $\alpha^3 = \alpha\alpha\alpha = (142)(142)(142) = (142)(124) = (1)(2)(4) = (1)$

        Then $|\alpha|$ = 3. Order of $\alpha$ is 3.

        $\beta$ = (13)\\
        $\beta^2 = (13)(13) = (1)$ \\
        Then $|\beta| = 2$
        
        \item \prop 8.10 Order of an r-cycle is $r$
        
        \item \exe 8.11 $\alpha = (143)(25)$\\
        $|\alpha| = LCM(|(143)|,|(25)|) = LCM(3,2) = 6 $

        \item \prop 8.12 Let $\alpha, \beta$ be two disgoint permutation. Then $|\alpha\beta| = LCM(|\alpha|,|\beta|)$
        
        \item Possible Disjoint Cycles
        
        \begin{tabular}{l|l|l|l|l}
            \hline
            Partition of 6 & Disjoint cycles & Example & Order & \# different permutations \\
            6 & 6 cycle & (132654)& 6 & $\frac{6!}{6} = 5!$ \\
            5 + 1& 5 cycle, 1 cycle & (13465)(2) & 5 & ${6 \choose 5} \frac{5!}{5} \frac{1!}{1} = {6 \choose 5}\cdot 4!$ \\
            4 + 2 & 4 cycle, 2 cycle & (1354)(26) & 4 & ${6 \choose 4}{2 \choose 2} \frac{4!}{4} \frac{2!}{2}$\\
            4 + 1 + 1 & (4,1,1) & (1354)(2)(6) & 4 & ${6 \choose 4}\frac{4!}{4}{2 \choose 1}\frac{2!}{2}{1 \choose 1}\frac{1!}{1} \frac{1}{2!}$
        \end{tabular}

        \textbf{NOTE}: We need to divide by the order since (123) = (231) = (312). We need to eliminate repeative terms.
        
        Also, we need to eliminate possible arrangement of cycles of the same length. In (4,1,1) the 1 cycles can appear in different orders but representing the same disjoint cycles.
        
        \textbf{Notation}: (6), (5,1), (4,2), (4,1,1), (3,3), (3,2,1), (3,1,1,1), (2,2,2), (2,2,1,1), (2,1,1,1,1), (1,1,1,1,1,1)
    \end{enumerate}
    
    \item 01.27 GROUPS!
    
    \begin{enumerate}
        \item \defi 9.11 G set 
        \begin{enumerate}
            \item $G \times G \rightarrow * G$ binary operation: $(x,y) \rightarrow x * y$
            \item associative law:
            
            $(x * y) * z= x*(y*z), \forall x,y,z \in G $

            \item $\exists e \in G $ is identity s.t. $e * x = x, x * e = x , \forall x \in G$
            \item $\forall x \in G, \exists y \in G s.t. x*y = e, y*x = e$ \\
            and $y$ is called inverse of $x$. (it is not necessarily unique)
            \item $x * y = y * x \forall x,y \in G$
        \end{enumerate}
        If only the \textbf{first 2} properties hold, it is called \textbf{semigroups}.

        If only the \textbf{first 3} properties hold, it is called \textbf{monoid}.

        If only the \textbf{first 4} properties hold, it is called \textbf{group}.

        If only the \textbf{all} properties hold, it is called \textbf{Commutative group (Abelian group)}.

        \item \textbf{Examples:}

        \begin{enumerate}
            \item $(\mathbb{Z}, + )$
            \begin{enumerate}
                \item $x,y \in \mathbb{Z}, x+y \in \mathbb{Z}$
                \item $(x+y)+z = x+(y+z)$
                \item $x + 0 = x, 0 + x = x, \forall x \in \mathbb{Z}$ therefore $e = 0$
                \item $x + y =0, y + x = 0 \rightarrow y = -x$
                \item $x + y = y + x$
                
                Then, $(\mathbb{Z}, + )$ is \textbf{Abelian group}
            \end{enumerate}
            \item $(\mathbb{Z}, \cdot )$
            \begin{enumerate}
                \item $x,y \in \mathbb{Z}, x\cdot y \in \mathbb{Z}$
                \item $(x\cdot y)\cdot z = x\cdot (y\cdot z)$
                \item $x \cdot 1 = x, 1 \cdot x = x, \forall x \in \mathbb{Z}$ therefore $e = 1$
                \item $x \cdot y =1, y \cdot x = 1 \rightarrow$ NO inverse in general. \{1,-1\} have inverse
                \item $x \cdot y = y \cdot x$
                
                Then, $(\mathbb{Z}, + )$ is a \textbf{commutative monoid} but not a \textbf{group}
            \end{enumerate}
            \item  $(\mathbb{Z}, - )$
            \begin{enumerate}
                \item $x,y \in \mathbb{Z}, x-y \in \mathbb{Z}$
                \item $(x-y)-z \ne x-(y-z)$ example: $2 - (1 - 5) \ne (2 - 1) - 5$
                
                Then, $(\mathbb{Z}, + )$ is not even a \textbf{semigroup}.\\
                We don't need to check following properties since it does not have an operation. All the following properties are target at the operation.
            \end{enumerate}
            \item  $(\mathbb{Z}_6, +_6)$
            \begin{enumerate}
                \item $[x]_6,[y]_6 \in \mathbb{Z}_6, [x]_6+[y]_6  = [x+y]_6\in \mathbb{Z}_6$
                \item $(x+y)+z = x+(y+z)$
                \item $e = [0]_6$
                \item inverse: $[-x]_6 + [x_6] = e$
                \item $[x]_6+[y]_6  =[y]_6+[x]_6 $
                
                $(\mathbb{Z}_6, +_6)$ is \textbf{Abelian group}
            \end{enumerate}
            \item  $(\mathbb{Z}_6, \cdot_6)$
            \begin{enumerate}
                \item $[x]_6,[y]_6 \in \mathbb{Z}_6, [x]_6\cdot[y]_6  = [x\cdot y]_6\in \mathbb{Z}_6$
                \item works
                \item $e = [1]_6$
                \item y does not always exists. only when $gcd(x,6) =1$ inverse exists.
                \item $[x]_6\cdot [y]_6  =[y]_6\cdot [x]_6 $
                
                $(\mathbb{Z}_6, +_6)$ is \textbf{commutative monoid} but not a \textbf{group}
            \end{enumerate}
            \item  $(\mathbb{Z}_6^{\times}, \cdot_6)$
            
            $\mathbb{Z}_6^{\times} = \{[x]_6 \in \mathbf{Z}_6 | gcd(x,6) = 1 \}$\\
            $\mathbb{Z}_6^{\times} = \{[1]_6, [5]_6 \}$
            \begin{enumerate}
                \item $[x]_6,[y]_6 \in \mathbb{Z}_6, [x]_6\cdot[y]_6  = [x\cdot y]_6\in \mathbb{Z}_6$
                \item works
                \item $e = [1]_6$
                \item holds!
                \item $[x]_6\cdot [y]_6  =[y]_6\cdot [x]_6 $
                
                $(\mathbb{Z}_6^{\times}, +_6)$ is \textbf{Abelian group}
            \end{enumerate}

            \item $(M_2(\mathbb{R}),+)$, $M_2(\mathbb{R}) = M \in \mathbb{R}^{2\times 2}$
            \begin{enumerate}
                \item Yes, there is a closed binary operation.
                \item associate law is inherted from +
                \item $e = \begin{bmatrix}
                    0 & 0 \\ 0 & 0
                \end{bmatrix}$
                \item inverses exist.
                \item commutative property holds.\\
                $(M_2(\mathbb{R}),+)$ is \textbf{Abelian group}
            \end{enumerate}

            \item $(M_2(\mathbb{R}),\cdot)$
            \begin{enumerate}
                \item Yes, there is a closed binary operation.
                \item (AB)C = A(BC)
                \item $e = \begin{bmatrix}
                    1 & 0 \\ 0 & 1
                \end{bmatrix}$
                \item inverses not necessaily exist. only $det(x) \ne 0$
                \item commutative property dose not hold.
                
                $(M_2(\mathbb{R}),+)$ is \textbf{monoid}
            \end{enumerate}

            \item $(GL_2(\mathbb{R}), \cdot)$ GL: general linear group -- determinants is $\ne$ 0
            \begin{enumerate}
                \item $det(AB) = det(A)det(B) \ne 0$
                \item (AB)C = A(BC)
                \item $e = \begin{bmatrix}
                    1 & 0 \\ 0 & 1
                \end{bmatrix}$
                \item inverse exists
                \item $AB \ne BA$ in general
                
                $(GL_2(\mathbb{R}), \cdot )$ is \textbf{Abelian group}
            \end{enumerate}

            \item $(S_3, \cdot)$\\
            $S_3 = \{(123),(132),(12),(13),(23),(1)\}$
            \begin{enumerate}
                \item $\alpha \cdot \beta = \alpha\beta$
                \item associative law good
                \item $e = (1)$
                \item inverse exists $(123)^{-1} = (321) \dots$
                % \item $AB \ne BA$ in general
                
                % $(GL_2(\mathbb{R}), \cdot )$ is \textbf{Abelian group}
            \end{enumerate}

        \end{enumerate}
    \end{enumerate} 
    \item 01.29
    \begin{enumerate}
        \item \recall 10.2
        
        monoids(prop 1,2,3) $\subseteq$ semigroups (prop 1,2) \\
        groups(prop 1,2,3,4) $\subseteq$ monoids(prop 1,2,3) \\
        Abelian groups(prop 1,2,3,4,5) $\subseteq$ groups(prop 1,2,3,4) \\

        \item 10.3 Let $G$ be a monoid, then $G$ has a unique identity element $e$
        \begin{proof}
            By definition of monoid, $\exists e \in G$ s.t $ex = x, xe = x, \forall x \in G$\\
            Suppose that $e$ and $e'$ are identity of $G$. i.e $ex = x, xe = x, e'x = x, xe' = x$\\
            WTS $e = e'$\\
            $e = e e' (e' \text{ is identity}) = e' (e \text{ is identity})$\\
            \textbf{NOTE:} we are using symmetric and transitive property of =
        \end{proof}

        \item \prop 10.4
        
        Leg $G$ be a group. Let $x \in G$ then $\exists ! y \in G \ \text{ s.t. } \ xy = e \text{ and } yx = e$

        \begin{proof}
            let $x \in G$ by definition, $\exists y \in G \text{ s.t. } xy = e, yx = e$

            Suppose $y \text{ and } y' \in G \text{ s.t. } xy = e, yx = e; xy' = e, y'x = e$\\
            WTS $y = y'$

            by assumption: $$xy = e$$\\
            operate $y'$ on the left: $$y'(xy) = y'e$$\\
            associate law: $$(y'x)y = y'e$$\\
            by assumption:$$y = y'e$$\\
            property of e:$$y = y'$$

            Therefore, $\exists ! y \in G \ \text{ s.t. } \ xy = e \text{ and } yx = e$
        \end{proof}
        
        \item \prop 10.5
        
        Let $G$ be a group then cancellation laws hold. i.e.

        $$ax = ay \Rightarrow x = y$$
        $$xa = ya \Rightarrow x = y$$

        \begin{proof}
            $$ax = ay$$
            Let $a^{-1}$ be the inverse of a, then operate $a^{-1}$ on both sides.
            $$a^{-1}(ax) = a^{-1}(ay)$$
            associative law:
            $$(a^{-1}a)x = (a^{-1}a)y $$
            property of inverse:
            $$ex = ey$$
            property of identity:
            $$x = y$$
            Therefore $ax = ay \Rightarrow x = y$\\
            similarly, $xa = ya \Rightarrow x = y$
        \end{proof}

        \item \defi 10.6 Subgroup
        
        Let $G$ be a group. A subgroup of $G$ is $H$ if:
        \begin{itemize}
            \item $H \subset G$ is a subset of $G$
            \item $H$ is a group under the same operation. i.e. ($H,*$) is a group
        \end{itemize}
        
        \item \exe 10.7 
        
        G = $(\mathbf{Z}, +)$, H = $(3\mathbf{Z}, +)$

        \item \exe 10.8
        
        G = $(\mathbf{Z}_6, +_6)$, H = $(3\mathbf{Z}, +)$

        H = $(\{[2], [4], [0]\}, +_6)$

        Need to show: 1. H is a subset 2. (H,$+_6$) is a group

        Just write a cayley table.

    \end{enumerate}

    \item 01.30
    
    \begin{enumerate}
        \item \defi 11.1 
        
        Let $(G, *)$ be a group, then $H$ is a subgroup of $G$ if 
        \begin{itemize}
            \item $H \subseteq G$, H is a subset of G
            \item $(H,*)$ is a group.
        \end{itemize}

        \item \exe 11.2
        
        $G = S_3 = \{(1), (123), (132), (12), (13), (23)\}$

        Operation - multiplication of permutation.

        $H = \{(1), (123), (132)\}$ is a subgroup of $G$

        $H$ is a proper subgroup of $G \Leftrightarrow H \ne G$
        \begin{itemize}
            \item $S_0$ $H\subset G$
            \item $S_{00}$ H is nonempty (i.e $H \ne \emptyset$) Usually, we check for identity.
            \item $S_1$ H is closed under operation
            \item $S_2$ H has inverse
        \end{itemize}
        \textbf{How To Check:}
        \begin{itemize}
            \item H is a subset of G: by def of elements of H
            \item H is a group under operation: 1. closed 2. associative 3. identity 4. inverse
        \end{itemize}

        When checking small sets, just use Cayley Table:

        \begin{tabular}{c|ccc}
            & (1) & (123) & (132) \\ \hline
            (1) & (1) & (123) & (132) \\ 
            (123) & (123) & (132) & (1) \\ 
            (132) & (132) & (1) & (123)
        \end{tabular}
        
        It is closed. Identity: e = (1). Inverse exists.
        
        \item \rem 11.3
        
        Let $(G,*)$ be a group then $(G,*)$ is a subgroup of itself.

        \item \defi 11.4
        
        Proper subgroup of $G$ is a subgroup $H$ s.t. $H\ne G$
    \end{enumerate}

    \item 02.03
    \begin{enumerate}
        \item \rem 12.2 We proved that if $G$ is a group, $x\in G$, then $\exists !$ inverse $y \in G$
        \item \prop 12.3 \\
        Leg $G$ be a group, let $x \in G$. If $xy = e$, then $y = x^{-1}$.\\ i.e. If $xy = e$, then $yx = e$
        \begin{itemize}
            \item since G is a group $\exists ! x^{-1}\in G$, multiply by $x^{-1}$ on the left. $$x^{-1}(xy) = x^{-1}e$$
            $x^{-1}(xy) = (x^{-1}x)y = ey = y = x^{-1}$
        \end{itemize} 

        \item 12.4 Restating 12.3: 
        
        Let $G$ be a group, $x\in G$, it is enough to check $xy=e$ to claim that $y = x^{-1}$

        \item 12.5 Let $G$ be a group. Suppose $x^n=e$ for some n postive integer. Then $x^{-1} = x^{n-1}$.
        \item \defi 12.6 Let $G$ be a group. Let $x \in G$ then order of x, denoted by $|x|$ is the \textbf{smallest positive integer} $n$ s.t. $x^n=e$. If such $n$ does not exist then $|x| = \infty$
        \item \defi 12.7 \\
        $G$ is a Group, $x \in G$.
        \begin{enumerate}
            \item $x^n := xxx \dots x $ (n-times) if n is positive integer.
            \item $x^0 := e$
            \item $x^{-1}:=$the inverse
            \item $x^{-n} := (x^{-1})^n = (x^n)^{-1}$
        \end{enumerate}
        Then $x^{n}$ is defined on $\mathbb{Z}$
        \item \prop 12.8 $G$ a group, $x \in G$ then\\
        $x^nx^m = x^{n+m}, \forall n,m \in \mathbb{Z}$
        \begin{proof}
            \begin{itemize}
                \item $n,m$ positive integer\\
                $x^nx^m = x\dots x$ (n-times) $x\dots x$ (m-times) = $x\dots x$(n+m times)\\
                Or use induction.
                \item $\dots$
            \end{itemize}
            Just use definition 12.7 to check all.
        \end{proof}
        \item $H$ is asubgroup of $G$ if
        \begin{itemize}
            \item $H$ is a subset of G
            \item $e\in H$ (check for not empty)
            \item $a,b \in H$ then $ab \in H$
            \item $a\in H$ then $a^{-1} \in H$
        \end{itemize}
        \item \rem 12.10 If $H$ is finite, then it is enough to check:
        \begin{itemize}
            \item $H$ is subset of $G$
            \item $H$ closed under operation
        \end{itemize} 
        \begin{proof}
            $x \in H$ then we have to take $x,x^1,x^2 \dots ,x^n\dots$
            since we are closed under operation.\\
            $H$ is finite $\Rightarrow \exists m,n$ s.t. $x^n=x^m$\\
            Suppose $n \ge m, x^n=x^m$,\\
            $$x^n = x^mx^{n-m} = x^m$$
            $$\Rightarrow x^mx^{n-m} = x^me$$
            $$\Rightarrow x^{n-m} = e$$
            Then it guarantees that there is identity in $H$
        \end{proof}

        \item \defi 12.12 $G$ a group, $x\in G$\\
        $<x>:=\{x^n|n\in \mathbb{Z}\}$
        
        \item \prop 12.13 $<x>$ is a subgroup of $G$
        \begin{proof}
            1. WTS $<x>\ \subset G$ 
            \begin{itemize}
                \item Positive power: $G$ is closed, then $x^2 = xx$ then $x^2$ is in $G$ By induction, $x^n$ is in $G$ for all positive n.
                \item 0 power: $x^0 = e$, $e$ is in $G$ since $G$ is a group.
                \item negative power: $x^{-1}$ is in $G$, then $x^{-n}$ is in $G$ (the same reason as the positive powers)
            \end{itemize}
            Then $<x> \ \subset G$ \\ \\
            2. WTS $a,b \in <x>,$ then $ab \in <x>$\\
            $a \in <x> \Rightarrow a = x^n, n \in \mathbb{Z}$\\
            $b \in <x> \Rightarrow b = x^m, m \in \mathbb{Z}$
            $\Rightarrow ab = x^nx^m = x^{n+m} \in <x>$\\ \\
            3.$a^{-1}$ in $<x>$
        \end{proof}
    \end{enumerate}

    \item 02.06
    
    \begin{enumerate}
        \item \defi 14.1 Let $G$ be a group, let $x \in G$, then conjugate of $x$ by $g\in G$ is $gxg^-1$
        \item \rem 14.2 
        \begin{itemize}
            \item $(13254)^{-1} = (45231)$
            \item $(13)^{-1} = (31) = (13)$ 
            \item $(ij)^{-1} = (ji)$
            \item $a^n = e\rightarrow a^{-1} = a^{n-1}$
        \end{itemize}
        \item \exe 14.3 $G = S_4$. Find a conjugate of $x = (143)$\\
        $g = (23)$ $gxg^{-1} = (23)(143)(23)^{-1} = (23)(143)(32) = (142)(3) = (142)$

        \item \exe Let $G= S_3$ Find all conjugate of $x = (13)$\\
        Since $S_3 = \{(1), (12), \dots\}$\\
        $(1)(13)(1)^{-1} = (13)$\\
        $(12)(13)(12)^{-1} = (23)$\\
        $(13)(13)(13)^{-1} = (13)$\\
        $(23)(13)(23)^{-1} = (12)$\\
        $(123)(13)(123)^{-1} = (12)$\\
        $(132)(13)(132)^{-1} = (23)$\\

        \item \prop 14.5 Let $\alpha \in S_n$, all conjugate of $\alpha$ = all permutations which have the none disjoint cycle decomposition as $\alpha$
        
        \item \exe 14.7 $G= S_5$ $\alpha = (142)(35)$ How many conjugates of $\alpha$ are there in $S_5$?\\
        i.e. How many permutations in $S_5$ has the form of (3-cycle)(2-cycle)? ${5 \choose 3}\frac{3!}{3} {2 \choose 2}\frac{2!}{2}$

        \item \prop Let $G$ be a group, Let $x \sim  y$ if $x$ is conjugate to y. Conjugate is an equivalece relation.
        
        \item \defi 14.9 Let $G$ be a group, $x \in G$, then conjugate(x):= $\{gxg^{-1} \ | \ g \in G\}$
        
        \item \rem 14.10 Conj. class of x = equivelence class of x under $y \sim z$ if y is conj to $z$
        
        \item \exe 14.11 Let $\alpha = (147)(235) \in S_8$. Find the size of conj class ($\alpha$).
        
        \item  $|\text{conj. class} (\alpha)|$ = ${8 \choose 3}\frac{3!}{3} {5 \choose 3}\frac{2!}{2} \frac{1}{2!}$
        
        \item \exe 14.12 $G = (\mathbb{Z}_4, +_4)$ Find all conj of $[2]_4$\\
        $[0] + [2] + [-0] = [2]$\\
        $[1] + [2] + [-1] = [1] + [2] + [3] = [2]$\\
        $[2] + [2] + [-2] = [2]$\\
        $[3] + [2] + [-3] = [2]$

        \item \prop 14.13 Let $G$ be an ebelian group. \\
        Conj. class(x) = \{x\} $|\text{Conj. class}(x)| = 1$ (property of commutative show that $gxg^{-1} = gg^{-1}x = ex = x)$

        \item Conj. Class(x) = \{x\} $\Rightarrow$ then $x$ commutes with all $g \in G$\\
        $G$ is disjoint union of its conj classes.

        \item \defi 14.15 Let $G$ be a group be $H$ be a subgroup of $G$. Let $g \in G$. Define $gHg^{-1}: \{ghg^{-1} \ | \ h \in H\}$ This is called conjugate of $H$ by g.

        \item \exe 14.16 Let $G = S_3$ let $H = \{(13), (1)\}$. Find conjugate of $H$ by $g = (123)$\\
        $(123)(13)(123)^{-1} = (123)(13)(321) = (1)(23)(321) = (12)$\\
        $(123)(1)(123)^{-1} = (1)$\\
        $gHg^{-1} = \{(12),(1)\}$

        \item \prop 14.17 Let $G$ be a goup, Let $H$ be a subgroup of $G$. Let $g \in G$ then $gHg^{-1}$ is a subgroup of $G$.
        \begin{proof}
            WTS: $gHg^{-1}$ is a subgroup of $G$\\
            0. $gHg^{-1}$ is a subset of $G$.\\
            00. $e \in gHg^{-1}$\\
            1. If $a, b \in gHg^{-1}$, then $ab \in gHg^{-1}$\\
            2. If $a \in gHg^{-1}$, then $a^{-1} \in gHg^{-1}$
            \begin{itemize}
                \item Show $gHg^{-1}$ is a subset of $G$.\\
                $\forall h \in H, g \in G, g, h, g^{-1} \in G$, $G$ is closed under operation.\\
                $\Rightarrow ghg^{-1} \in G$. Then $gHg$ is a subset of $G$.
                \item Show $e \in gHg^{-1}$\\
                Since $H$ is a group, $\Rightarrow e \in H$.\\ $geg^{-1} = e \Rightarrow e \in gHg^{-1}$.
                \item Show If $a, b \in gHg^{-1}$, then $ab \in gHg^{-1}$\\
                $\exists a', b' \in H$ such that $ga'g^{-1} = a, gb'g^{-1} = b$.\\
                $ab = ga'g^{-1}gb'g^{-1} = ga'(g^{-1}g)b'g^{-1} = g(a'b')g^{-1}$\\
                Therefore, $ab \in gHg^{-1}$
                \item Show If $a \in gHg^{-1}$, then $a^{-1} \in gHg^{-1}$\\
                $\exists a' \in H$ such that $ga'g^{-1} = a$\\
                Since $H$ is a group, $a'^{-1} \in H$\\
                Then, $ga'^{-1}g^{-1} \in gHg^{-1}$.\\
                $a\cdot ga'^{-1}g^{-1} \\
                = ga'g^{-1}ga'^{-1}g^{-1} \\
                = ga'(g^{-1}g)a'^{-1}g^{-1} \\
                = ga'ea'^{-1}g^{-1} \\
                = ga'a'^{-1}g^{-1} \\
                = g(a'a'^{-1})g^{-1} \\
                = geg^{-1}  \\
                = gg^{-1} \\
                = e$\\
                Therefore, $a^{-1} = ga'^{-1}g^{-1} \in gHg^{-1}$
            \end{itemize}
            
        \end{proof}
        
    \end{enumerate}
    \item 02.10 
    \textbf{ ISOMORPHISMS of GROUPS}
    \begin{enumerate}
        \item \defi 15.1 Let $G$, $G'$ be groups. A function $f: G\rightarrow G'$ is called isomorphisms of groups if:
        \begin{itemize}
            \item $f(x*y) = f(x)*f(y)$
            \item $f$ is one to one (injective)
            \item $f$ is onto (surjective)
        \end{itemize}
        \item \rem 15.2\\
        First multiply and then apply $f$ is the same as first apply $f$ and then multiply
        \item \exe 15.3\\
        $G = (S_2, \cdot) = \{(1),(12)\}, G' = (Z_2, +_2) = \{[0]_2, [1]_2\}$\\
        Then $f: (1) \rightarrow [0]_2, (12) \rightarrow [1]_2$\\
        We need to check:\\
        $f((1)\cdot(1)) =? f(1)+_2f(1)$\\
        Check: $f((1)\cdot(1)) = f((1)) = [0]_2$\\
        $f(1)+_2f(1) = [0]_2 + [0]_2 = [0]_2 $\\
        $f((1)\cdot(12)) =? f(1)+_2f(12)$\\
        Check: $f((1)\cdot(12)) = f((12)) = [1]_2$\\
        $f(1)+_2f(12) = [0]_2 + [1]_2 = [1]_2 $\\
        $f((12)\cdot(12)) =? f(12)+_2f(12)$\\
        Check: $f((12)\cdot(12)) = f((1)) = [0]_2$\\
        $f(12)+_2f(12) = [1]_2 + [1]_2 = [0]_2 $\\
        $f((12)\cdot(1)) =? f(12)+_2f(1)$\\
        Check: $f((12)\cdot(1)) = f((12)) = [1]_2$\\
        $f(12)+_2f(1) = [1]_2 + [0]_2 = [1]_2 $\\
        \textbf{For small groups, just use cayley tables}\\
        \begin{tabular}{c|cc}
            $S_2$ & (1) & (2)\\ \hline
            (1) & (1) & (12)\\
            (12) & (12) & (1)
        \end{tabular}, 
        \begin{tabular}{c|cc}
            $Z_2$ & 0 & 1\\ \hline
            0 & 0 & 1\\
            1 & 1 & 0
        \end{tabular}\\
        If the cayley tables are the same under the mapping $f$, then it is isomorphism.
        \item \exe 15.4 $G= (S_2, \cdot)$, $G'= (Z_2, +_2)$ consider the mapping: $(1) \rightarrow [1]_2, (12)\rightarrow [0]_2$\\
        Property 2, 3 is trivial, Check the property one.\\
        $f((1)(12)) = f((12)) = [0]_2$. while $f(1)+f(12) = [1]_2+ [0]_2 = [1]_2$\\
        Therefore, this mapping is not a isomorphism.
        \item \exe 15.5 $G= (Z_4, +_4)$, $G'= (Z^{\times}_10, *_10)$\\
        $G = \{[0], [1], [2], [3]\}$, $G'=\{1,3,7,9\}$\\
        \textbf{NOTE:} $G = <[1]_1>$, $G' = <3>$. \\
        Therefore, a possible way to define a map is $[1]_4^i \rightarrow 3^i$.\\
        i.e. $[1] \rightarrow 3$ \\
        $[1]^2 = [2]\rightarrow 3^2 = 9$ \\
        $[1]^3=[3] \rightarrow 3^3 = 7$ \\
        $[1]^4 = [0]\rightarrow 3^4 = 1 $
        \item \defi 15.6 Let $G$ be a group, If $G$ is finite then order of $G = |G|$ is defined to be the number of elements in $G$. If $G$ is not finite, then $|G| = \infty$ 
        \item \prop 15.7 \ Suppose $G\cong  G'$ i.e. $ \exists f:G\rightarrow G'$ that is isomorphic. Then $|G| = |G'|$.
        \item \prop 15.8 \ Suppose $G\cong  G'$ Then $G$ is abelian $\Leftrightarrow$ $G'$ is abelian. (which implies that abelian group and non abelian group is not isomorphic)
        \item \defi 15.10 A group $G$ is cyclic if $G = <a>$ for some $a \in G$
        \item \rem 15.11 $(Z_4, +_4)$ is cyclic, $(Z_{10}^{\times}, \cdot_{10})$ is cyclic
        \item \prop 15.12 if $G$ and $G'$ are cyclic and $|G| = |G'|$ then $G\cong G'$ 
        \begin{proof}
            idea of proof: \\
            G cyclic, $G = <a>$, G' cyclic, $G' =<a'>$ \\
            Then $f: G \rightarrow G'$: $a^i \rightarrow (a')^i$\\
            Then check f is bijection.
        \end{proof}
        \item \prop 15.13 Let $G = (Z_n, +_n) = \{[0], [1] \cdots, [n-1]\}$, $G' = (Z_n, +_n) = (\{0,1,2,\cdots, n-1\}, +_n)$
        Then $G\cong G'$ and the isomorphisim can be take $[x]_n \rightarrow x$

    \end{enumerate}

    \item 02.12
    \begin{enumerate}
        \item \defi 16.1 Order of a group: \# of elements in that group.
        \item \defi 16.2 Order of an element: smallest positive integer n such that $g^n = e$
        \item \prop 16.3 Let $g \in G$, $|<g>| = |g|$\\
        \begin{proof}
            $<g> = \{\dots, g^{-1}, g^0, g^1, g^2, \dots\}$\\
            $|g| = n \Rightarrow g^n = e$ n is the smallest positive integer.\\
            Then $<g> = \{e, g, g^2 \dots, g^{n-1}\}$\\
            \textbf{CLAIM:} they are all distinct elements\\
            \textbf{proof:} 
            suppose $g^i = g^j$ for some $0 \le i,j \le n-1$\\
            Suppose $i \le j$\\
            $g^{-i} g^i = g^{-i}g^j$\\
            $e = g^{j-i}$\\
            Since $i \le j$, then $j-i \ge 0$\\
            Therefore, $j - i = 0$\\
            Therefore, $j = i$\\
            Therefore, $g^i = g^j$\\
            Therefore, for all $0,1,2,\dots,n-1$, $g^i$ are all distinct.\\
            Therefore, $g^i$ are all distinct.\\
            Therefore, $|<g>| = n \Rightarrow |<g>| = |g|$ 
        \end{proof}
        \item \rem 16.4 Let $g\in G$ then $<g>$ is a subgroup of $G$.
        \item \rem 16.5 $<g>$ is cyclic group (is a cyclic subgroup of G)
        \item \theo 16.6 Let $H$ be a subgroup of group $G$ then $|H|$ divides $|G|$
        \item \exe 16.7 Let $G$ be a group with $G$ = 7, Then the only subgroups of $G$ are $H = G$ or $H = \{e\}$\\
        If $H < G$ is a subgroup of $G$ then $|H|$ divides $|G|$\\
        Therefore, $|H| = 1 $ or 7.\\
        Therefore, $H = G$ or $H = \{e\}$
        \item \coro 16.8 $g \in G$, then $|g|$ divides $|G|$
        \begin{proof}
            $|g|=|<g>|$, $<g>$ is a subgroup of $|G|$.\\
            Then $|<g>|$ divides $|G|$.\\
            Then $|g|$ divides $|G|$.
        \end{proof}
        \item Let $G= S_3$ let $x \in G$ what are the possible orders of x?\\
        $|G| = |S_3| = 6$\\
        $\Rightarrow |x| = 1,2,3,6$
        We know that orders are only 1,2,3. Not all divisors appear as order of elements.
        \item \prop 16.10 Let $G$ be a group, $|G| = n < \infty$ then $G$ is cyclic $\Leftrightarrow$ $\exists x \in G, |x| = n$
        \begin{proof}
            \textbf{Exercise}

        \end{proof}
        \item 16.11 $G=S_3$ $G$ is not cyclic since there is no element of order $|G|$.
        \item \rem 16.12 $<g> = \{g^i\}$\\
        Under addition: $(G,+)$ then $<g> = \{0,g,2g,3g,\dots, ng\}$
        \item \recall 16.14\\
        $G = (Z_12, +_12)$ then $G = \{4,8,12 = 0\}$
        \item \exe 16.14 $G = (Z_12, +_12)$, show, $G$ is cyclic
        \begin{proof}
            To show $G$ is cyclic, we need to find an element of order 12.\\
            1 has order 12 since $<1> = \{1,2,3,4,5,6,7,8,9,10,11,12 = 0 = e\}$\\
            Then we can say: \textbf{1 is a generator of G}\\
            Any number that is relatively prime to 12 is a generator of $G$\\
            1,5,7,11 are generators of $G$
        \end{proof}
        \item \defi 16.15 Let $G$ be a group. Let $H$ be a subgroup. Let $a \in G$. a \textbf{left coset} of $H$ in $G$
        $$aH := \{ah \ | \ h \in H\}$$
        \textbf{NOTE: Not simply mean multiplication. It is the binary operation in the group}
        \item \exe 16.16 Let $G = S_3$ Let $H = <(13)> = \{(13), (1)\}$. Find all left cosets of $H$\\
        \begin{tabular}{c|cc}
            & (13) & (1)\\ \hline
            (1) & (13) & (1)\\
            (12) & (132) & (12)\\
            (13) & (1) & (13)\\
            (23) & (123) & (23)\\
            (123) & (23) & (123)\\
            (132) & (12) & (132)\\
        \end{tabular}
        \begin{itemize}
            \item Disjoint (or the same) cosets
            \item aH = H $\Leftrightarrow$ $a \in H$
            \item $a \in aH$
            \item $a \in bH \Leftrightarrow aH = bH \Leftrightarrow b \in aH$
            \item $|H| = |aH|$
            \item \# of disjoint cosets = 3 = $\frac{|G|}{|H|}$
            \item $G = \cup $ all cosets.
        \end{itemize}
    \end{enumerate}

    \item 02.13
    
    \begin{enumerate}
        \item \rem 17.1 If H is a subgroup of G, then $|H| \ | \ |G|$ ($|H|$ divides $|G|$)
        \item \prop 17.3
        \begin{itemize}
            \item All subgroup of $Z$ are cyclic
            \item All subgroup of $Z_n$ are cyclic
        \end{itemize}
        \item \prop 17.4 Let $G = Z_n$, k|n then $|<k>| = |k| = \frac{n}{k}$
        \item \prop 17.5 Let $G = Z_n$, k|n then $|<k>| = |k| = |gcd(n,k)| =  \frac{n}{gcd(n,k)}$
        \item \prop 17.6 Let $G = Z_n$, $d|n$ then $\exists!$ subgroup of order d. It is given as $k = \frac{n}{d}$, $|<k>| = d$
    \end{enumerate}

+1 (01.27)
\end{enumerate}
\end{document}