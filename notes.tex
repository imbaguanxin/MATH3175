\documentclass[12pt]{article}  
\usepackage{graphicx}
\usepackage{geometry}   %设置页边距的宏包
\usepackage{algpseudocode}
\usepackage{comment}
\usepackage{amsmath, amssymb, amsthm}
\usepackage{enumerate}
\usepackage{enumitem}
\usepackage{framed}
\usepackage{verbatim}
\usepackage{microtype}
\usepackage{kpfonts}
\usepackage{multicol}
\usepackage{amsfonts}
\usepackage{array}
\usepackage{color}
\usepackage{pgf,tikz}
\usetikzlibrary{automata, positioning, arrows}
\usepackage{wrapfig}
\newcommand{\solu}{{\color{blue} Solution:}}
\newcommand{\theo}{{\color{blue} Theorem: $\ $}}
\newcommand{\defi}{{\color{blue} Definition: $\ $}}
\newcommand{\recall}{{\color{blue} Recall: $\ $}}
\newcommand{\exe}{{\color{yellow} Exercise: $\ $}}
\newcommand{\prop}{{\color{blue} Prop: $\ $}}

\newcommand{\hw}{{\color{red} Homework: $\ $}}
\newcommand{\overbar}[1]{\mkern 1.5mu\overline{\mkern-1.5mu#1\mkern-1.5mu}\mkern 1.5mu}
\newcommand{\Ib}{\mathbf{I}}
\newcommand{\Pb}{\mathbf{P}}
\newcommand{\Qb}{\mathbf{Q}}
\newcommand{\Rb}{\mathbf{R}}
\newcommand{\Nb}{\mathbf{N}}
\newcommand{\Fb}{\mathbf{F}}
\newcommand{\Z}{\mathbf{Z}}
\newcommand{\Lap}{\mathcal{L}}
\newcommand{\Zplus}{\mathbf{Z}^+}
\geometry{left=2cm,right=2cm,top=1.5cm,bottom=2cm}  %设置 上、左、下、右 页边距

% \tikzset{
%     ->, % makes the edges directed
%     >=stealth, % makes the arrow heads bold
%     node distance=2.5cm, % specifies the minimum distance between two nodes. Change if necessary.
%     every state/.style={thick, fill=gray!10}, % sets the properties for each ’state’ node
%     initial text=$ $, % sets the text that appears on the start arrow
% } 

\title{CS3800 HW2}
\date{}
\author{Xin Guan}


\begin{document}  
\begin{enumerate}
    \item 01.22
    \begin{enumerate}
        \item \defi 7.1
        
        Function $f: X \rightarrow Y$ is bijection if $f$ is both surjection(on to) and injection (one to one)

        \item \theo 7.2 
        
        $f: X \rightarrow Y$ is bijection $\Leftrightarrow$\\
        $\exists g: Y \rightarrow X$ s.t. $g \circ f = id_x, f \circ g = id_y$ ($id_x$ means identity)

        Such $g$ is called the inverse of f. Denoted by $f^{-1}$

        \item \recall 
        
        $\circ$ Composition of two injective functions is injective. \\
        $\circ$ Composition of two surjective functions is surjective.\\
        $\circ$ Composition of two bijective functions is bijective.\\

        \item \defi 7.4 Permutation:
        
        Permutation on set $X$ is a bijection $f : X \rightarrow X$

        \item prop 7.5
        \begin{enumerate}
            \item if $f: X \rightarrow X$ is a permutation then $\exists f^{-1}: X \rightarrow X$ which is also permutation.
            \item composition of two permutation is again a permutation.
        \end{enumerate}

        \item \defi 7.6
        
        if $X = \{1,2,\dots,n\}$ then, $S_n :=$ \{all permutation on $X$\} 

        \item EX 7.7
        
        $\alpha = (1 \ 2 \ 3 \ 4 \ 5) \rightarrow (3 \ 4 \ 1 \ 2 \ 5)$\\
        $\beta = (1 \ 2 \ 3 \ 4 \ 5) \rightarrow (5 \ 1 \ 4 \ 3 \ 2)$

        Find $\alpha\beta$ (composition of $\alpha$ and $\beta$), $\alpha^{-1}$

        \solu 

        $(\alpha\beta)(1) = \alpha(\beta(1)) = \alpha(5) = 5$\\
        $(\alpha\beta)(2) = \alpha(\beta(2)) = \alpha(1) = 3$\\
        $(\alpha\beta)(3) = \alpha(\beta(3)) = \alpha(4) = 2$\\
        $(\alpha\beta)(4) = \alpha(\beta(4)) = \alpha(5) = 1$\\
        $(\alpha\beta)(5) = \alpha(\beta(5)) = \alpha(2) = 4$

        Then, $\alpha\beta = (1 \ 2 \ 3 \ 4 \ 5) \rightarrow (5 \ 3 \ 2 \ 1 \ 4)$\\

        $\alpha^{-1} = (3 \ 4 \ 1 \ 2 \ 5) \rightarrow (1 \ 2 \ 3 \ 4 \ 5)$\\
        rearrange:\\
        $\alpha^{-1} = (1 \ 2 \ 3 \ 4 \ 5) \rightarrow (3 \ 4 \ 1 \ 2 \ 5)$

        \item \hw 2.1 9(b) 
        
        $g : \Z_8 \Rightarrow \Z_12, g([x]_8) = [6x]_12$ show that g is well defined.
        
        \solu

        \begin{proof}
        Suppose $[x]_8 = [x']_8$, WTS $g([x]_8) = g([x']_8)$

        Let $[x]_8 = [x']_8 \\ \Rightarrow x \equiv x' (mod \ 8) \\ \Rightarrow 8|(x - x') \\ \Rightarrow x - x' = 8*q $ for some $q \in \Z$ \\
        $x = 8\cdot q + x'$

        By definition of $g$, $g([x]_8) = [6x]_{12}$\\
        Then, $g([x]_8) = [6(8q+x')]_{12} = [48q + 6x']_{12}, g([x']_8) = [6x']_{12}$\\
        WTS $[48q + 6x']_{12} = [6x']_{12}$\\
        Enough to show: $12 | (48q + 6x' - 6x')$\\
        Since $48q + 6x' - 6x' = 48 q = 12 \cdot 4 \cdot q$\\
        $\Rightarrow 12 | 12 \cdot 4 \cdot q$\\
        $\Rightarrow 12 | (48q + 6x' - 6x')$ \\
        $\Rightarrow g([x]_8) = g([x']_8)$
        \end{proof}
        
        

    \end{enumerate}
\end{enumerate}
\end{document}