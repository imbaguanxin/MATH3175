\documentclass[12pt]{article}  
\usepackage{graphicx}
\usepackage{geometry}   %设置页边距的宏包
\usepackage{algpseudocode}
\usepackage{comment}
\usepackage{amsmath, amssymb, amsthm}
\usepackage{enumerate}
\usepackage{enumitem}
\usepackage{framed}
\usepackage{verbatim}
\usepackage{microtype}
\usepackage{kpfonts}
\usepackage{multicol}
\usepackage{amsfonts}
\usepackage{array}
\usepackage{color}
\usepackage{pgf,tikz}
\usetikzlibrary{automata, positioning, arrows}
\usepackage{wrapfig}
\newcommand{\solu}{{\color{blue} Solution:}}
\newcommand{\theo}{{\color{blue} Theorem: $\ $}}
\newcommand{\defi}{{\color{blue} Definition: $\ $}}
\newcommand{\recall}{{\color{blue} Recall: $\ $}}
\newcommand{\exe}{{\color{green} Example: $\ $}}
\newcommand{\prop}{{\color{blue} Prop: $\ $}}
\newcommand{\rem}{{\color{blue} Remark: $\ $}}


\newcommand{\hw}{{\color{red} Homework: $\ $}}
\newcommand{\overbar}[1]{\mkern 1.5mu\overline{\mkern-1.5mu#1\mkern-1.5mu}\mkern 1.5mu}
\newcommand{\Ib}{\mathbf{I}}
\newcommand{\Pb}{\mathbf{P}}
\newcommand{\Qb}{\mathbf{Q}}
\newcommand{\Rb}{\mathbf{R}}
\newcommand{\Nb}{\mathbf{N}}
\newcommand{\Fb}{\mathbf{F}}
\newcommand{\Z}{\mathbf{Z}}
\newcommand{\Lap}{\mathcal{L}}
\newcommand{\Zplus}{\mathbf{Z}^+}
\geometry{left=2cm,right=2cm,top=1.5cm,bottom=2cm}  %设置 上、左、下、右 页边距

% \tikzset{
%     ->, % makes the edges directed
%     >=stealth, % makes the arrow heads bold
%     node distance=2.5cm, % specifies the minimum distance between two nodes. Change if necessary.
%     every state/.style={thick, fill=gray!10}, % sets the properties for each ’state’ node
%     initial text=$ $, % sets the text that appears on the start arrow
% } 

\title{MATH 3175 Notes}
\date{}
\author{Xin Guan}


\begin{document}  
\maketitle
\begin{enumerate}
    \item 01.22
    \begin{enumerate}
        \item \defi 7.1
        
        Function $f: X \rightarrow Y$ is bijection if $f$ is both surjection(on to) and injection (one to one)

        \item \theo 7.2 
        
        $f: X \rightarrow Y$ is bijection $\Leftrightarrow$\\
        $\exists g: Y \rightarrow X$ s.t. $g \circ f = id_x, f \circ g = id_y$ ($id_x$ means identity)

        Such $g$ is called the inverse of f. Denoted by $f^{-1}$

        \item \recall 
        
        $\circ$ Composition of two injective functions is injective. 

        $\circ$ Composition of two surjective functions is surjective.

        $\circ$ Composition of two bijective functions is bijective.

        \item \defi 7.4 Permutation:
        
        Permutation on set $X$ is a bijection $f : X \rightarrow X$

        \item prop 7.5
        \begin{enumerate}
            \item if $f: X \rightarrow X$ is a permutation then $\exists f^{-1}: X \rightarrow X$ which is also permutation.
            \item composition of two permutation is again a permutation.
        \end{enumerate}

        \item \defi 7.6
        
        if $X = \{1,2,\dots,n\}$ then, $S_n :=$ \{all permutation on $X$\} 

        \item EX 7.7
        
        $\alpha = \begin{pmatrix}
            1 & 2 & 3 & 4 & 5 \\
            3 & 4 & 1 & 2 & 5
        \end{pmatrix}$\\
        $\beta = \begin{pmatrix}
            1 & 2 & 3 & 4 & 5 \\
            5 & 1 & 4 & 3 & 2
        \end{pmatrix}$

        Find $\alpha\beta$ (composition of $\alpha$ and $\beta$), $\alpha^{-1}$

        \solu 

        $(\alpha\beta)(1) = \alpha(\beta(1)) = \alpha(5) = 5$\\
        $(\alpha\beta)(2) = \alpha(\beta(2)) = \alpha(1) = 3$\\
        $(\alpha\beta)(3) = \alpha(\beta(3)) = \alpha(4) = 2$\\
        $(\alpha\beta)(4) = \alpha(\beta(4)) = \alpha(5) = 1$\\
        $(\alpha\beta)(5) = \alpha(\beta(5)) = \alpha(2) = 4$

        Then, $\alpha\beta = \begin{pmatrix}
            1 & 2 & 3 & 4 & 5 \\
            5 & 3 & 2 & 1 & 4
        \end{pmatrix}$

        $\alpha^{-1} = \begin{pmatrix}
            3 & 4 & 1 & 2 & 5 \\
            1 & 2 & 3 & 4 & 5
        \end{pmatrix}$\\
        rearrange:\\
        $\alpha^{-1} = \begin{pmatrix}
            1 & 2 & 3 & 4 & 5 \\
            3 & 4 & 1 & 2 & 5
        \end{pmatrix}$

        \item \hw 2.1 9(b) 
        
        $g : \Z_8 \Rightarrow \Z_12, g([x]_8) = [6x]_12$ show that g is well defined.
        
        \solu

        \begin{proof}
        Suppose $[x]_8 = [x']_8$, WTS $g([x]_8) = g([x']_8)$

        Let $[x]_8 = [x']_8 \\ \Rightarrow x \equiv x' (mod \ 8) \\ \Rightarrow 8|(x - x') \\ \Rightarrow x - x' = 8*q $ for some $q \in \Z$ \\
        $x = 8\cdot q + x'$

        By definition of $g$, $g([x]_8) = [6x]_{12}$\\
        Then, $g([x]_8) = [6(8q+x')]_{12} = [48q + 6x']_{12}, g([x']_8) = [6x']_{12}$\\
        WTS $[48q + 6x']_{12} = [6x']_{12}$\\
        Enough to show: $12 | (48q + 6x' - 6x')$\\
        Since $48q + 6x' - 6x' = 48 q = 12 \cdot 4 \cdot q$\\
        $\Rightarrow 12 | 12 \cdot 4 \cdot q$\\
        $\Rightarrow 12 | (48q + 6x' - 6x')$ \\
        $\Rightarrow g([x]_8) = g([x']_8)$
        \end{proof}
    \end{enumerate}
    \item 01.23
    \begin{enumerate}
        \item \recall
    
        DEF: Permutation on set X is a bijection $f: X \rightarrow X$

        NOTE: $S_x$ = \{permutation on X\}, $S_n$ = \{permutation on \{1,2,3, ... n\}\}

        PROPERTIES: 
        
        composition of permutation is again a permutation.

        identity map: $id: X \rightarrow X (id(x) = x)$ is a permutation.

        each permutation $f$ there is an inverse $f^{-1}$ such that $f \circ f^{-1} = id, f^{-1} \circ f = id$.

    \item \defi 8.1 Disjoint cycle decomposition
    
    Suppose $\alpha = \begin{pmatrix}
        1 & 2 & 3 & 4 & 5 & 6 & 7 & 8 \\
        3 & 4 & 8 & 1 & 2 & 6 & 7 & 5
    \end{pmatrix}$ \\= (1 3 8 5 2 4)(6)(7) or (1 3 8 5 2 4) (in cycle notation)
    
    \item \defi 8.2 
    
    2-cycle $\rightarrow$ (i j) $i\ne j$ \\
    3-cycle $\rightarrow$ (i j k) i,j,k distinct \\
    r-cycle $\rightarrow$ ($i_1, i_2, \dots, i_r$),  $i_1, i_2, \dots, i_r$ distinct\\

    \item \exe 8.3
    
    $\alpha = (142), \beta = (13)$
    $\alpha \rightarrow$ 3-cycle, $\beta \rightarrow$ 2-cycle.

    \item identity permutation in $S_n$
    
    \begin{enumerate}
        \item (1)(2)$\dots$(n)
        
        \item fixes $\forall i$
        
        \item 1-cycle $(i)$ fixes $i$
        
        \item ofthen we donot note 1-cycle: $\alpha = (142) = (142)(3)$
        
        \item id = (1) = (1)(2)$\dots$(n)
    \end{enumerate}

    \item \exe 8.5 nultiplication of permutation
    
    $\alpha = (142), \beta = (13), \in S_4$

    compute - write as a product of disjoint cycles (same as \exe 7.7 with new notation)

    $\alpha\beta = (142)(13) = (1342)$

    HOWTO: $\beta: 1 \rightarrow 3$, then $\alpha 3 \rightarrow 3$, then (13) now.\\
    $\beta 3 \rightarrow 1$, then $\alpha 1 \rightarrow 4$, then (134) now.\\
    $\beta 4 \rightarrow 4$, then $\alpha 4 \rightarrow 2$, then (1342).

    Similarly: $\beta\alpha = (13)(142) = (1423)$

    \item \rem 8.6
    Ingeneral $\alpha\beta \ne \beta\alpha$\\
    if $\alpha, \beta$ are disjoint then $\alpha\beta  = \beta\alpha$
    \end{enumerate}

    \item \defi 8.7 
    
    Order of permutation $\alpha$ is the smallest positive integer $n$ such that $\alpha^n = (1)$ where $\alpha^n = \alpha \alpha \dots \alpha$ (there are n $\alpha$'s)

    \item \exe 8.8 

    $\alpha = (142)$

    $\alpha^2 = \alpha \alpha = (142)(142) = (124)$

    $\alpha^3 = \alpha\alpha\alpha = (142)(142)(142) = (142)(124) = (1)(2)(4) = (1)$

    Then $|\alpha|$ = 3. Order of $\alpha$ is 3.

    $\beta$ = (13)\\
    $\beta^2 = (13)(13) = (1)$ \\
    Then $|\beta| = 2$
    
    \item \prop 8.10 Order of an r-cycle is $r$
    
    \item \exe 8.11 $\alpha = (143)(25)$\\
    $|\alpha| = LCM(|(143)|,|(25)|) = LCM(3,2) = 6 $

    \item \prop 8.12 Let $\alpha, \beta$ be two disgoint permutation. Then $|\alpha\beta| = LCM(|\alpha|,|\beta|)$
    
    \item Possible Disjoint Cycles\\
    \begin{tabular}{l|l|l|l|l}
        \hline
        Partition of 6 & Disjoint cycles & Example & Order & How many different permutation \\
        6 & 6 cycle & (132654)& 6 & $\frac{6!}{6} = 5!$ \\
        5 + 1& 5 cycle, 1 cycle & (13465)(2) & 5 & ${6 \choose 5} \frac{5!}{5} \frac{1!}{1} = {6 \choose 5}\cdot 4!$ \\
        4 + 2 & 4 cycle, 2 cycle & (1354)(26) & 4 & ${6 \choose 4}{2 \choose 2} \frac{4!}{4} \frac{2!}{2}$
    \end{tabular}
    NOTE: We need to divide by the order since (123) = (231) = (312). We need to eliminate repeative terms.
\end{enumerate}
\end{document}